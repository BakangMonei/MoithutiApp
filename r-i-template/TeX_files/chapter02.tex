\chapter{BACKGROUND LITERATURE}


\section{LITERATURE REVIEW}
In the last decade we have seen tremendous growth in the use of mobile phones by not only the public but by students in the higher sector, with students being able to navigate through the device and use them to their advantage. The immense availability of mobile devices offers an opportunity for the students to integrate into the educational sector by looking up different varsity constituted. In one application system that gives them an opportunity to apply for courses and sponsorship at once. To come up with a solidified proof that indeed the poor admission system is a challenge, a Questionnaire was conducted and response from intended audience. According to (CITE)‘In the absence of a ‘unified’ or ‘Cluster system’ of admission test, students must buy forms of different universities. Not being certain of getting admissions to the university they want to study in.’ when we compare the above statement to what is currently happening with the application process of tertiary institutions, the same problem is faced. It is not by doubt that a unified system which is being brought forth will save the students the worry and time of filling multiple forms manually. Infact, an opportunity of having to submit their applications to more than one institution will increase their chances of being accepted. Although some institutions have their own websites it is very important to acknowledge that most of the websites are not student friendly based on the fast that they are difficult to navigate, and the only solution is a unified student application system.

Online-Registration Systems 
Several registration systems are in use in Botswana's universities and colleges, some of which support online registration and others do not. The THITO of BAC and UB is one example. Some of these systems are developed internally by software development teams in each university or college's computer center (IT). This registration system differs from others in that it is an ALL-IN-ONE Registration System. First and foremost, the system is explicitly used to assist students in applying to multiple schools through a single system, as well as applying for sponsorship.(CITE)\\

COMMON ONLINE APPLICATION SYSTEM
The Online applications system are intended to accord anyone seeking admission into University of Botswana and BAC study programmes to apply online provided they have an email account. Prospective applicants may use any Internet facility available to them to apply. It is important for each and every prospective applicant to carefully read all admission requirements and other related documents enclosed as A Guide to Prospective Applicants under “STUDY” before s/he applies for admission. With the BAC application system prospective applicants get to click the browse Catalogue button.(CITE)
The use of this system makes the chances of applicant admission to a college higher as there is a presented opportunity to apply to multiple universities at a time. The use of an online system where forms are filled and that same forms being sent to all selected universities in just a single click save students the time of having to fill multiple forms manually and submitting physically.In Botswana students or prospective applicants do not have the privilege and opportunity of searching for schools based on their major in the common application. The students applying for the university admission will have a particular course they would want to study. So, if they are given the chance to search for universities of their choice based on the course they would like to offer, it automatically cancels the risk of applying to a university that does not take an intended course of study. There should also be a comprehensive information about each member institution such as a link to an institution official website. This will help students with a fair idea of the institution they are applying to.
As a result, if they are given the opportunity to search for universities of their choice based on the course they wish to offer, the risk of applying to a university that does not offer the desired course of study is eliminated. There should also be detailed information about each member institution, such as a link to the official website of the institution. This will give students a good idea of the institution to which they are applying.

THE NEED FOR A UNIFIED STUDENT APPLICATION MANAGEMENT SYSTEM
In a fast-growing, global country, an effective and efficient educational system is essential. Students can perform better when educational institutions are prepared to meet global and international standards. Establishing a student application management system is the first step toward reaching this goal.(CITE)
When universities and colleges in Botswana and around the world adopt student application management systems, a lot more can be archived and executed with less effort. One major advantage of student applications is that admissions processes become simpler and faster. Universities receive hundreds of thousands of applications from aspiring students almost every year or at the start of the academic year. Unfortunately, not every student is able to make the cut. Administering a rigorous legal vetting of documents and evaluating process can be time consuming and difficult to apply consistently since most colleges and universities do not have online application systems and are forced to do things manually. With ever-increasing competition, new tools and technologies are developed almost daily to help universities raise their visibility and increase applicant conversion.  

ADVANTAGES OF STUDENT APLLICATION MANAGEMENT SYSTEM
The use of a student application management system aids in cost reduction. Previously, students had to spend more money on making copies of their transcripts as well as traveling to prestigious schools to collect the forms. With the implementation of the student application management system, evaluations will be computerized and faster. This eliminates the need to spend money on printing materials or files that would then be used to transfer data from one unit to another. Electronic storage is also more efficient and environmentally friendly. This, in turn, has a positive implication for financing. Using an application management system saves management money that would otherwise be spent on mishandling and incorrect processing.
Another advantage of a student application management system is its accuracy. Errors with names and numbers, as well as mishandling of documents, may occur during manual document collation. Using a student application management system reduces errors because an AI tool verifies authenticity and consistency. When a mistake occurs because of incorrect information or unverifiable details, students are given feedback. (cite)

A student application management system provides educational institutions with a global reach and outlook. In addition to processing and verifying applicant information, systems are programmed to assist educational institutions in monitoring market growth and development. Every entry into the registration portal can be tracked and quantified in order to account for where and how applicants learn about the HEI. As a result, data such as the number of applicants can be tracked.(cite)More than anything, making the admission process more accessible to prospective students helps an institution to have more applicants and, ultimately, more enrolments. As globalization shrinks the distance between places and people, now is the time for HEIs to focus on growing. The strongest tool to ensure this growth is a robust student application management system.
INFORMATION SECURITY
Most campus management system tools developed for higher education management are intended to provide complete security of students' information. When applying for a course of study and a tertiary institution, students must provide personal information, financial information, and other vital information. However, before sharing this information, students must ensure that it is secure.

When it comes to higher education management, it is critical to understand that they are dealing with young adults who are cautious about the information they are sharing. If the institution requires students to share their financial credit scores, they must provide adequate support and persuade them that the information is secure. As a result, one of the most significant benefits of a college management system is that the information of the students is safe and secure. (cite)
INFORMATION UPDATE
There is no need to worry about updating information while using a college management system. This is something that can be done automatically and even in real time. There is no need to keep separate hard copies of all information because it can be stored in a centralized database.
Once these are updated, all that is required is to enter the information into the centralized system, and the system will automatically be updated with the most recent information. If there are concerns about software updates, they are also performed automatically. Usually, a software-based college management system is a service provided by a third-party.

As a result, the company providing Software-as-a-Service, or SaaS, will update the software automatically. There is no need to worry about updating information or software. As a result, college management software can be extremely beneficial in higher education. It assists the college administration in gaining a thorough understanding of the institute's operations and assists students with their studies through the use of e-learning tools. Using a software-based college management system will also help to alleviate the administrative burden on your college's administrative staff and simplify the entire college management process.(cite)

THE NEED TO APPLY TO MORE THAN ONE COLLEGE/UNIVERSITIES\\
After applicants have visited colleges and narrowed down the list of schools to which they are interested, it is time to decide where they will apply. There is no right or wrong number of applications to submit, whether two or seven, but applying to multiple colleges is strongly advised for a variety of reasons. With little knowledge of what is available, students tend to apply to the one or two tertiary institutions that they believe they can get into, feel can afford, or simply feel comfortable attending for reasons such as knowing someone else who is already enrolled and attends the same tertiary institution. The less they know about admissions and affordability, the more benefit there is in applying to a more diverse list of colleges so that they can get a clearer picture of their true options as they progress through the application, admissions, and financial aid processes. That doesn't mean they should throw a ball of colleges and apply to the ones they land on. Community colleges, universities, technical colleges offering two-year degree programs or6 months programs, private colleges, colleges that are close to home, and colleges out of the village can all be on an applicant list. Take note that not applying, means no admission into institution of choice. (cite)
Applying for college has become much more competitive in the past two decades.
Just because an applicant meets an institutions minimum entrance criteria does not guarantee you an acceptance letter. Applying to multiple schools increases your chances of acceptance.
Apply to the colleges an applicant wants to attend as well as at least one safety school.
Most experts recommend applying to a couple of reach institutions, several tertiary institutions that a prospective applicant believe are a good academic and social fit and a couple of safety schools.
What happens if an applicant applies to one dream school and one safety school that they are not thrilled about, and they are not accepted into their first-choice college? Attend the safety school that was only a backup plan in your mind? If they are unsure of where they stand academically, perhaps because they earned good grades but attend a large high school where it is difficult to determine where you truly rank, applying to six or seven colleges with varying entrance criteria and different student body numbers may provide you with multiple options to determine best fit, which may even surprise them. However, keep in mind that most college applications require a fee. A good example is the University of Botswana and the Institute of Health Science. Applying to schools where you know you won't meet the basic entrance requirements can cost you money. It is best to save that money and put it toward college costs if you are accepted.
Chances are that an applicant has specific need that only few colleges meet
If only a few colleges fit an applicant specific need, then they do not have to apply to a ton of colleges. If they want to be in a very specific location or pursue a major that only a few schools have, then you may only need to apply to 2 or 3 colleges. What they need to do is look up the course catalogue to see which school is best when it comes to offering the course of choice

