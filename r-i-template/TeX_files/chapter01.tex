\chapter{INTRODUCTION}

\section{BACKGROUND STUDY}
Student admissions are an important part of any institution's operation because prospective students are what keep a university or a college running. Admission is one of the most important and crucial operations to be carried out in an institution because a university or college cannot exist without students. It is critical to understand that a poor admissions system means that only a small number of students are admitted to the university or college of their choice, or that this harbours an excessively slow response time. Admission processes are critical for both the institution and the candidate.

The admission rate in Botswana continues to increase as more student’s progress from high school to tertiary institutions. By the academic year 2014/15, it was calculated that 25 825 learners were enrolled with colleges having a share of 39.2\% and Universities having a share of 37.9\%. With a slight decrease in the transition rate of the academic year 2017/18 compared to academic year 2016/17 of 9.5\%. Enrolment in 2017/18 was standing at 59,243 with students admitted at public universities and 11 299 in private institutions. Studies have shown that there is an increase as the numbers continue to rise from 47,889 to 53 450 students admitted. Likewise, a lower transition rate indicates a low share of students progressing due to various reasons such as limited intake capacity of the tertiary education system.\cite{}

Botswana has lost a lot of potential talent because of the strict selections caused by the availability of spaces in varsity. The situation is exacerbated further by the scarcity of available accommodations. Unfortunately, tertiary institutions cannot even admit a quarter of applicants due to insufficient facilities. Worse, the government's economic recession makes it difficult for students to even get sponsorship, even if they meet the university's set requirements. Statistics show that the year 2016/17 the government was able to give full sponsorship to 38 806 Batswana students. Based on the statistics by year 2016/17, 30.19\% of students managed to qualify and gain admissions into the university of their choice. The question that remains is what happens to the 69.81\% remaining is anyone’s guess. This project is directed towards that 69.81\% that struggle to get admissions.\cite{}

Currently, the admission process in Botswana requires prospectus students to fill out admissions forms online or even apply in person at schools. The issue is that if a student wants to apply to more than one school of choice, they must visit the schools to obtain admission letters and then submit them, which appears to be costly to them. The effectiveness of this method will be discussed as the project progresses.

\section{PROJECT SIGNIFICANCE}
The purpose of this project is to make tertiary admission application process a more conducive and easier one. With this project its aim is to design a system where our prospectus students can apply once to several institutions as well as have the privilege of applying for sponsorship.
\subsection{PROBLEM STATEMENT}
The COVID-19 pandemic has forced many education and training institutions around the globe to switch from traditional face-to-face applications to online methods. With pandemic still at hand it impacted the world at large we have seen many activities being halted and being stopped, and one great example is a career fair in Botswana. Many students are approaching educational consultants to get admission through Management quota. They take the admission recommended by friends or neighbours. They directly get the admission as they don’t have enough time to enquire about different colleges. The transition continues to pose further challenges for developing countries such as Botswana in terms of the preparedness of training systems and the availability of digital technologies for online applications. Off current in Botswana usage of new technology in schools is very much limited. This has continued to pose as a major concern as young people continue to face high demands in a society which is characterized by a quick rate of change in the labor market due to the dormancy of globalization, use of technology in education system, increased dissemination of information, and numerous opportunities that continue to avail themselves. In today’s world making a career choice is seen as a crucial phase in life and young people have a great responsibility for their own future career but at the same time the schools in general often lack strategies to increase young people’s knowledge and to support them in their choices regarding their futures. Despite this challenge, students can apply to as many institutions as they want provided, they qualify. By this they must fill each form belonging to the tertiary institution, some schools even require students to pay for admission fee even when they are not sure of being admitted. This is time consuming and costly. With this much information gathered we evaluate the impact of limited technology and poor admission system at hand.

\subsection{PROPOSED SOLUTION}
The key solution to avoiding all the problems mentioned is to find a unified way that solves the problem. The only unified way is by computerization. The intended solution brought forth by the system to the current system that involves filling forms manually for different institutions, is the use of a common application which will allow for students to fill admission forms once and apply to multiple institutions of choice.

\subsection{AIMS AND OBJECTIVES}
\subsubsection{AIM}
The aim of this project is to bring all tertiary institution applications in Botswana to a common interface. In addition to increasing the likelihood of prospective students being admitted to a tertiary institution.

\subsubsection{OBJECTIVES}
\begin{itemize}
	\item To allow for filling of application letter and sponsorship letter
	\item Give users opportunity to submit applications to multiple institutions at once.
	\item determine whether is feasible to develop the new system
	\item identify the users of the system.
	\item Efficient and safe storage
	\item Quick retrieval of records and information
	\item Quick response to any enquires
	\item Assist in admission updating

\end{itemize}

\section{PROJECT OVERVIEW}
The project that will be developed is a mobile application system for all universities and colleges. In general, the prospective student will use the system to apply. A mobile application system is software that is beneficial to both students and tertiary institutions, because in the current system all process is done manually, and this necessitates a significant amount of time and manpower. All these issues will be addressed through the use of a mobile application system, which is designed to automate students' ability to navigate the system and find all relevant information about the course of their choice, as well as the ability to apply to their desired schools and receive sponsorship, thereby reducing paperwork. Almost all work is computerized in the system, and accuracy is easily maintained. All details and information about a student, tertiary institution, financial aid, and course catalogue can be handled by the system administrator. The details include tertiary details, course details, student personal details and to mention a few. It is the administrator's responsibility to insert, update, and monitor the entire process that will take place and all data is securely stored on a SQL Lite and Firebase database that ensures the highest level of security possible. 

\section{SCOPE AND LIMITATION}
\subsubsection{PROJECT SCOPE}
This document presents a detailed description of the Moithuti Application system. It explains Features of the system, what the system will do. The project deals with various functionalities in application process, with an idea of implementing a proper system that can be used now and in the near future. Our system contains many operations such as opening account, applying for courses, applying for sponsorship. Within the system design/ model it deals with a student applying for a course from schools of their choice as well as sponsorship by filling in the forms and attaching relevant documents which are then stored in the database and display of student’s details is retrieved from the database tables.

\subsubsection{PROJECT LIMITATIONS}

\section{REQUIREMENT}
The goal of this report is to analyze the functional and non-functional needs of users in order to develop and continuously improve the Moithuti Mobile App, which is absolutely essential. The information provided will be based on the functional requirement specifications found on the Moithuti mobile application app. Our requirements were gathered by obtaining admission forms for Botswana's tertiary institutions as well as online applications. We know what common information tertiary institutions require when filling out forms based on these admissions.

\subsubsection{REQUIREMENT SPECIFICATION}
The system we decided to create is called Moithuti Mobile application. The goal of this project is to create a system that would act like an automated version of the manual student application system, with the goal of reducing human work while retaining accuracy. The system we are creating is designated to have four access mode or actors which is the User, Institution, Financial aid, and the system administrator. The administrator will be managing the system and be accountable for the whole process. To have access to the system, the application should be free to download from a mobile store provided there is access to the internet. The proposed system will provide users with relevant courses to register for along with the course catalogue. For the system to do so, it allows for the user to have an account, which means they must firstly log in if they already have an account or sign up if they do not.

To save time of the prospective users and to have information at the tip of their finger, the system will include a layer with various tertiary institutions along with respective courses offered categorised, which the user will navigate through at their own comfort time and get obtain information about the course that they desire and know where their course of interest is offered. The Moithuti mobile application will provide a rapid, dependable, and user-friendly mechanism for students to use when applying for a course of their choice and institution, with the system having a form that requires them to attach relevant documents such as their transcripts and a copy of their national identity card. 

Once the institution has reviewed the application the approved prospective student will be sent admission via their email address that they would have provided when filling in the application form. The system is also convenient and effective in a way that it will allow approved students to apply for government sponsorship, but for self-sponsored students their process will end with them having received their admission letter. For those applying for government sponsorship the system will also have a layer that would allow them to fill in the application form and attach necessary documents.

The system will allow all applications forms filled in by the prospective students to be saved securely in a database and only the administrator will be able to access it. The database should also entail of the prospective student’s full names, email addresses, cell number and passwords. The tertiary institution details and course details.  

The system will be programmed in such a way that it can be used any day and at any time to prevent inconveniencing the users who will desire to use the system at their preferred time. This will also ensure that the Moithuti mobile application will attract more users thus helping a lot of prospective students to avoid the tedious process of manually applying for tertiary institution.

\subsubsection{FUNCTIONAL REQUIREMENTS}
The intended Moithuti mobile app has 4 core functionalities that will play the most impact of the system to the users. They are listed below:
\begin{itemize}
	\item User must be able to download the Moithuti mobile application from Playstore.
	\item User can log in to the software or sign up.
	\item User must have a device that has the capacity to access internet.
	\item User should be able to change password.
	\item User should be able to use the application as a search directory to derive information about institutions.
	\item User should be able to manage their applications.
	\item Administrator should have a real time update on the system.
	
\end{itemize}

These functional requirements will define how the Moithuti mobile app will function.

\begin{enumerate}
	\item User Login
	Home page entails of the:
	Search button
	\item Courses category content
	\item Institution’s content
	\item Email content
	\item Campus information
	\item About the institution
	\item Sponsorship application content
	\item Contact details of each institution
	\item Terms and conditions
	\item Moithuti Application privacy agreement
	\item User Logout
\end{enumerate}

\subsubsection{NON FUNCTIONAL REQUIREMENTS}
\begin{itemize}
	\item The system should be readily available for download.
	\item The system performance and response time must be fast for instant information.
	\item The system should have a layer of robustness which means that the stored documents and login credentials will always be available and accessible.
	\item Application should be able to satisfy all user requirements mobility functions.
	
\end{itemize}

\subsubsection{REQUIREMENTS MASTER LIST}
\begin{table}[ht]
	\centering
	\begin{tabular}{|c|p{7cm}|p{7cm}|}
		\hline
		ID NO. & FUNCTIONAL REQUIREMENTS & NON FUNCTIONAL REQUIREMENTS\\
		\hline
		Req1 & Student must download the Moithuti mobile application to
		their device.&Application shall be readily available to download from Android Google Playstore.\\
		\hline
		Req2 & Student can install the system into a phone. & System should be available for download.\\
		\hline
		Req3& the user must have a phone that is connected to the internet enable the user to access the online application database.& The system should be fast and responsive.\\
		\hline
		Req4& Student must sign up to create account and get access to the app. & Account creation should be easy and straightforward to execute.\\
		\hline
		Req5 & Student clicks the sign-Up button.&Button redirects user to the sign-up form.\\
		\hline
		Req 6 &Student should fill the sign in form. &Form should be clear to understand and easy to use.\\
		\hline
		Req 7&Student clicks the register button &The button redirects student to home page.\\
		\hline
		Req 8& Student must login to the system.& The login process should be fast.\\
		\hline
		Req 9&The system then validates the login details. &The application should save credentials of the accounts.\\
		\hline
		Req10&The student receives a login successful message. & A quick pop up message appears to the user appears which is easy and clear to read\\
		\hline
		Req11 &Student should be able to reset password. &The system should allow students to change passwords.\\
		\hline
		Req12&User who already have accounts can click forgot password button. &System should provide the user with the forgot password button to enable them to retrieve their account.\\
		\hline
		Req13 & After successful login, Student must be able to navigate through the software.& Navigation should be easy.\\
		\hline
		Req14 &User should be able to deactivate account & System should allow user to deactivate account\\
		\hline
		Req15 & Student must be able to view the home page.&System should provide on-the-go information.\\
		\hline
		Req16 & Student must be able to view the course catalogue.&System should be user friendly.\\
		\hline
		Req17 & Student must be able to view course catalogue in real-time. & The system shall update the course catalogue.\\
		\hline
		
	\end{tabular}
\end{table}

\begin{table}[ht]
	\centering
	\begin{tabular}{|c|p{7cm}|p{7cm}|}
		\hline
		ID NO. & FUNCTIONAL REQUIREMENTS & NON FUNCTIONAL REQUIREMENTS\\
		\hline
		Req18 & Student can click institution button.& Button redirects to the institutions page.\\
		\hline
		Req19&Student should be able to click course button for more information. &System button should respond in time for instant information should be at the fingertips of users.\\
		\hline
		Req20 & Student should have access to the About page of each institution. & Application should satisfy mobility.\\
		\hline
		Req21 & Student should have access to the contacts of institutions in cases of clarity.& Application should satisfy versatility.\\
		\hline
		Req22 & Student can click the apply here button to register for their desired course.& The button redirects the user to the institution Application form.\\
		\hline
		Req23 &The student should fill in application form. & The form should be convenient and easy to fill.\\
		\hline
		Req24 & Student should be able to attach necessary documents. & System should allow attachment of files and documents.\\
		\hline
		Req25 & Student should be able to click send the submit button to send their application.& The submit button should be responsive and fast.\\
		\hline
		Req26 &User can click cancel button to cancel the application. & System should provide the user with an option to navigate back.\\
		\hline
		Req27 & The student should receive the 
		‘Application successful’ message.
		& A quick pop-up message appears to the user.\\
		\hline
		Req28 & Student must be able to retrieve their application from the system. & System should provide security backup incase the data gets lost or damaged.\\
		\hline
		Req29 & Student should be able to go back to a certain step in cases where the device shuts down. & System should save all the progress of the user before abnormal happening.\\ 
		\hline
		Req30 & Student should be able retrieve and submit course application to administrator’s email. & System should give access to the administrator to receive application emails.\\
		\hline
		Req31& Student shall have an option button to apply for sponsorship. & The button will redirect the student to the application form.\\
		\hline
		Req32 & The student shall fill the application form.
		& System should allow the form to be edited.\\
		\hline
		Req33 & Student should be able to attach required files on the application form. & The system must provide a paperclip link to indicate that files can be added.\\
		\hline
		Req34 & System can be accessed by the student and administrator only. & System should provide Security and Privacy Policy.\\
		\hline
		Req35 & Student should be able to access the application anytime, any day. & System should be reliable and constantly available. \\
		\hline
		Req36 & Administrator information should have real time update. & The system should be able to backup data.\\
		\hline
		Req37 & The system allows user to log out& System must have a log out button.\\
		\hline
		
	\end{tabular} 
\end{table}


